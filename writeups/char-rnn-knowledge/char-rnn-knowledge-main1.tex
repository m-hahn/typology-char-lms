% File tacl2018.tex
% Aug 3, 2018

% The English content of this file was modified from various *ACL instructions
% by Lillian Lee and Kristina Toutanova
%
% LaTeXery is all adapted from acl2018.sty.


% To check:
% - Submission must be in A4 format
% - min length 7 pages, max length 10 of CONTENT

% Example table:

% \begin{table}[t]
% \begin{center}
% \begin{tabular}{|l|rl|}
% \hline \bf Type of Text & \bf Size & \bf Style \\ \hline
% paper title & 15 pt & bold \\
% \iftaclfinal
% author names & 12 pt & bold \\
% author affiliation & 12 pt & \\
% \else
% \fi
% the word ``Abstract'' as header & 12 pt & bold \\
% abstract text & 10 pt & \\
% section titles & 12 pt & bold \\
% document text & 11 pt  &\\
% captions & 10 pt & \\
% %bibliography & 10 pt & \\
% footnotes & 9 pt & \\
% \hline
% \end{tabular}
% \end{center}
% \caption{\label{tab:font-table} Font requirements}
% \end{table}


\documentclass[11pt,a4paper]{article}
\usepackage[hyperref]{tacl2018v2} % use ``nohyperref'' to disable  hyperref
\usepackage{times,latexsym}
\usepackage{url}
\usepackage[T1]{fontenc}

%\taclfinalfalse % For camera-ready, replace "\taclfinalfalse" with
% "\taclfinalcopy"

%%%%
%%%% Material in this block can be removed by TACL authors.
% It consists of things specific to generating TACL instructions
\usepackage{xspace,mfirstuc,tabulary}


% packages added by Marco and Michael
\usepackage{paralist} 
\usepackage{graphicx} 
\usepackage{multirow} 
\usepackage{enumitem}
\usepackage{linguex}
%\raggedbottom

%\newcommand{\ex}[1]{{\sf #1}}


\title{\emph{Tabula} nearly \emph{rasa:} Probing the linguistic knowledge of character-level neural language models trained on unsegmented text}


% The command \taclfinalfalse suppresses display of the contents of the
% \author{...} command in the generated pdf.
% Replacing that command with "\taclfinalcopy" reveals the author info in the
% generated pdf.
% See tacl2018.sty for other ways to set author info.
\author{
 Template Author\Thanks{The {\em actual} contributors to this instruction
 document and corresponding template file are given in Section
 \ref{sec:contributors}.} \\
 Template Affiliation/Address Line 1 \\
 Template Affiliation/Address Line 2 \\
 Template Affiliation/Address Line 2 \\
  {\sf template.email@sampledomain.com} \\
}

\date{}

\begin{document}
\maketitle
\begin{abstract}
  Recurrent neural networks (RNNs) have reached striking performance in
  many natural language processing tasks. This has renewed interest in
  whether these generic sequence processing devices are inducing
  genuine linguistic knowledge. Nearly all current analytical studies,
  however, initialize the RNNs with a vocabulary of known words, and
  feed them tokenized input during training. We present a
  multi-lingual study of the linguistic knowledge encoded in RNNs
  trained as character-level language models, on input data with word
  boundaries removed. These networks face a tougher and more
  cognitively realistic task, having to discover and store any useful
  linguistic unit from scratch, based on input statistics. The results
  show that our ``near \emph{tabula rasa}'' RNNs are mostly able to
  solve morphological, syntactic and semantic tasks that intuitively
  presuppose word-level knowledge, and indeed they learned to track
  ``soft'' word boundaries. Our study opens the door to speculations
  about the necessity of an explicit word lexicon in language learning and
  usage.
\end{abstract}


\section{Introduction}
\label{sec:introduction}


Recurrent neural networks \cite[RNNs,][]{Elman:1990}, in particular
in their Long-Short-Term-Memory variant
\cite[LSTMs,][]{Hochreiter:Schmidhuber:1997}, are the current
workhorse of natural language processing. RNNs, often
pre-trained on the simple \emph{language modeling} objective of
predicting the next symbol in natural text, are a crucial
component of state-of-the-art architectures for machine
translation, natural language inference and text categorization
\cite{Goldberg:2017}.

RNNs are very general devices for sequence processing, hardly assuming
any prior linguistic knowledge. Moreover, the simple prediction task
they are trained on in language modeling is well-attuned to the core
role prediction plays in cognition
\cite[e.g.,][]{Bar:2007,Clark:2016}. RNNs have thus long attracted
researchers interested in language acquisition and processing. Their
recent successes in large-scale tasks has rekindled
this interest \cite[e.g.,][]{Frank:etal:2013,Lau:etal:2017,Kirov:Cotterell:2018,Linzen:etal:2018,McCoy:etal:2018,Pater:2018}.

The standard pre-processing pipeline of modern RNNs assumes that the
input has been tokenized into word units that are pre-stored in the
RNN vocabulary. This is a reasonable practical approach, but it makes
simulations less interesting from a linguistic point of view. First,
discovering words (or other primitive constituents of linguistic structure) is one of the major challenges a learner faces, and
by pre-encoding them in the RNN we are facilitating its task in an
unnatural way (not even the staunchest nativists would take specific
word dictionaries to be part of our genetic code). Second, assuming a
unique tokenization into a finite number of discrete word units is in
any case problematic. The very notion of what counts as a word in
languages with a rich morphology is far from clear
\cite[e.g.,][]{Bickel:Zuniga:2017}, and, universally, lexical knowledge
is probably organized into a not-necessarily-consistent hierarchy of
units at different levels: morphemes, words, compounds, constructions,
etc.~\cite[e.g.,][]{Goldberg:2005}. Indeed, it has been suggested that
the notion of word cannot even be meaningfully defined
cross-linguistically \cite{Haspelmath:2011}.

Motivated by these considerations, we study here RNNs that are trained
without any notion of word units in their input or in their
architecture. We train our RNNs as \emph{character-level neural
  language models}
\cite[CNLMs,][]{Mikolov:etal:2011,Sutskever:etal:2011,DBLP:journals/corr/Graves13}
by removing whitespace from their input, so that, like children
learning a language, they don't have access to explicit cues to
wordhood.\footnote{We do not erase punctuation marks, reasoning that
  they have a similar function to prosodic cues in spoken language.}
This setup is almost as \emph{tabula rasa} as it gets. By using
unsegmented orthographic input (and assuming that, in the alphabetic
writing systems we work with, there is a reasonable correspondence
between letters and phonetic segments), we are only postulating that
the learner figured out how to segment the continuous speech stream
into phonological units, an ability children already possess few
months after birth \cite[e.g.,][]{Maye:etal:2002,Kuhl:2004}.

After training the networks on the unsupervised character-level
language modeling task, we test them on a bank of linguistic tests in
German, Italian and English. We focus on these languages due to
resource availability and ease of benchmark construction. However, we
also argue that well-studied synthetic languages with a clear,
orthographically-driven notion of word constitute a natural starting
point to test non-word-centric models, compared to agglutinative or
polysynthetic languages, where the very notion of what counts as a
word is problematic. % While one of
% our ultimate goals is precisely to study how word-less models process
% languages whose grammatical system is less clearly word-based,
% starting with languages in whose analysis the orthographic word has
% traditionally played a central role is a reasonable ``sanity check''.
  
Our tasks intuitively require a model to have developed a latent
ability to parse characters into word-like items associated to
morphological, syntactic and broadly semantic features. The RNNs
pass most of the tests, suggesting that they are in some way able to
construct and manipulate the right lexical items. In a final experiment,
we look more directly into \emph{how} the models are handling
word-like units. We find, confirming an earlier observation by
\newcite{Kementchedjhieva:Lopez:2018}, that the RNNs specialized some
units to the task of detecting word boundaries (or, more generally,
salient linguistic boundaries, in a sense to be further discussed
below). Taken together, our results suggest that character-level RNNs
capture various forms of linguistic knowledge that is intuitively
word-based, without being exposed to an explicit segmentation of their
 input and, more importantly, without possessing an explicit
lexicon. We will discuss the implications of these findings in the
conclusion.\footnote{Upon publication, we will make our input data,
  test sets and pre-trained model available.}

% probe them with phonological, lexical,
% morphological, syntactic and semantic tests in English, German and
% Italian. Our results show that near-\emph{tabula-rasa} CNLMs acquire
% an impressive spectrum of linguistic knowledge at various levels.
% This in turn suggests that, given abundant input (large Wikipedia
% dumps), a learning device whose only prior architectural bias consists
% in the LSTM memory cell implicitly acquires a variety of linguistic
% rules that one would intuitively expect to require much more prior
% knowledge.


\section{Related work}
\label{sec:related}

\paragraph{Character-based neural language models} Character-level
RNN-based language models have received some attention in the last
decade because of their potential greater generality with respect to
word-based models, as well because, intuitively, they should be able
to use cues, such as morphological information, that word-based models
miss by design. Early studies such as \newcite{Mikolov:etal:2011},
\newcite{Sutskever:etal:2011} and \newcite{Graves:2014} established
that CNLMs are in general not as good at language modeling as their
word-based counterparts, but lag only slightly behind (note that
character-level sentence prediction involves a much larger search
space than predicting at the word
level). \newcite{Sutskever:etal:2011} and \newcite{Graves:2014}
presented informal qualitative analysis showing that the CNLMs are
learning basic linguistic properties of their input. The latter, who
trained LSTM models, also showed that CNLMs can keep track to some
extent of hierarchical structure, as showing by their ability to
correctly balance parentheses when generating text. Our aim here is to
understand to what extent CNLMs trained on unsegmented input learn
various linguistic constructs. This is different from much work in the
area, that has focused on \emph{character-aware} architectures, that
combine in various ways character- and word-level information, with
the goal to develop state-of-the-art language models, also effective
in morphologically rich languages \citep[see, e.g.,][and references
there]{Bojanowski:etal:2016,Kim:etal:2016,Gerz:etal:2018}. \textbf{Give
  example of difference between character-level and character-aware.}
Note that all earlier work, even when not assuming an \emph{a priori}
word vocabulary, trains on text with whitespace and other orthographic
word-boundary cues. There is work on (morpheme-level) segmentation
using character-level RNNs \cite{Kann:etal:2016}, but the emphasis
there is in optimizing segmentation, as opposed to our interest in
probing what the network implicitly learned about morphemes and other
units.


Earlier work by Jeffrey Elman is close in spirit to ours. In
particular, \newcite{Elman:1990} presented experiments on phonotactics
and word segmentation related to ours, but using small-size toy
input. More recently, \cite{Radford:etal:2017} study CNLMs (or, more
precisely, byte-level models) with focus on understanding their
properties, but they focus on sentiment analysis. Closest to us,
\cite{Alishahi:etal:2017} study the linguistic knowledge induced by a
neural network that receives unsegmented acoustic input. They use
however a considerably more complex architecture, trained on
multimodal data, and focus on the phonological level. \textbf{Discuss
\newcite{Godin:etal:2018}.}

\paragraph{Probing linguistic knowledge of neural language models} There is extensive work on probing the linguistic properties of
word-based neural language models, as well as more complex
architectures such as sequence-to-sequence systems, see, e.g.,
\newcite{Li:etal:2016,Linzen:etal:2016,Shi:etal:2016,Adi:etal:2017,Belinkov:etal:2017,Hupkes:etal:2017,Kadar:etal:2017,Li:etal:2017}. Closest
to us, \newcite{Sennrich:2017} tests character- and subword-unit-level
models used as components of a machine translation system on a variety
of grammatical tests. He concludes that current character-based
decoders generalize better to unseen words, but capture less
grammatical knowledge than subword units. Still, his character-based
systems lag only marginal behind the subword architectures on
grammatical tasks such as handling agreement and negation.








\section{Experimental setup}
\label{sec:setup}

Describe source corpus for each language, and how it was
pre-processed. Training/validation/test partitioning.

Hyperparameter search. LSTM vs RNN. Word-level model (was it also
optimized?)

Report results in terms of BPS on test partition, with ballpark
estimate of where we are in terms of the state of the art.



\section{Experiments}
\label{sec:experiments}

\subsection{Discovering phonotactic constraints}
\label{sec:phonotactics}

Focusing on German and Italian, that have reasonably transparent
orthographies.

How setup here differs from general one.

Method: construct pairs of letter bigrams (corresponding to phoneme
bigrams) beginning with the same letter, such that one is
phonotactically acceptable in the language and the other isn't, but
the independent unigram probability of the unacceptable bigram is
higher than that of the acceptable one. E.g., ``\emph{br}'' is
acceptable Italian sequence, ``\emph{bt}'' isn't, although
\emph{``t''} is more frequent. We re-train the CNLM on a version of
the corpus from which both bigrams have been removed. We then look at
the likelihood the model assigns to both sequences. If the model
assigns a larger probability to the correct sequence, it means that it
implicitly possesses a notion of phonological categories such as
stops and sonorants, which allows it to correctly generalize from
attested (e.g., ``\emph{tr}'') sequences to unattested ones
(\emph{``br''}).

LSTM vs RNN

Results, in a table with all pairs for both languages, as in Table
\ref{tab:phonotactics-results}.

\begin{table}[t]
  \begin{center}
    \begin{tabular}{ll|cc}
      \multicolumn{2}{c}{\emph{bigrams}}&\emph{LSTM}&\emph{RNN}\\
      \hline
      \multicolumn{4}{c}{\emph{German}}\\
      \hline
      \ldots & \ldots & \ldots & \ldots \\
      \ldots & \ldots & \ldots & \ldots \\
      \hline
      \multicolumn{4}{c}{\emph{Italian}}\\
      \hline
      pa & pb & \ldots & \ldots \\
      \ldots & \ldots & \ldots & \ldots \\
    \end{tabular}
  \end{center}
  \caption{\label{tab:phonotactics-results} Figure of merit could be
    log likelihood ratio between acceptable and unacceptable bigram,
    with positive values in bold.}
\end{table}


Discussion: note that generalization of model are purely
distributional, with no aid from perceptual or articulatory cues.


\subsection{Word segmentation}
\label{sec:segmentation}

English/German/Italian

Specifics of this setup, and differences from general setup

Does the model develop an implicit notion of word?

Should we use only one method here? One that is unsupervised, for
direct comparison to the Bayesian approach? We could then report that,
when training on the hidden state, we get super-high accuracy,
indicating that the information is there.

LSTM vs RNN

Results could be summarized as in Table \ref{tab:segmentation-results}.


\begin{table}[t]
  \begin{center}
    \begin{tabular}{l|lll|lll|lll}
      \multicolumn{1}{c}{}&\multicolumn{3}{c}{\emph{LSTM}}&\multicolumn{3}{c}{\emph{RNN}}&\multicolumn{3}{c}{\emph{Bayes}}\\
      \hline
      &P&R&F&P&R&F&P&R&F\\
      \hline
      English &\ldots & \ldots & \ldots &\ldots & \ldots & \ldots&\ldots & \ldots & \ldots\\
      German &\ldots & \ldots & \ldots &\ldots & \ldots & \ldots&\ldots & \ldots & \ldots\\
      Italian &\ldots & \ldots & \ldots &\ldots & \ldots & \ldots&\ldots & \ldots & \ldots\\
    \end{tabular}
  \end{center}
  \caption{\label{tab:segmentation-results} Obviously, we must reduce space between P, R and F}
\end{table}

Report syntactic depth plot: illustrates how it is useful for
segmentation knowledge to be implicit, as the model ``knows'' about
different kinds of boundaries in a continuous manner.

Qualitative analysis: report proportions of over- and
under-segmentation for our best CNLM, look at common over- and
under-segmentation errors in English? E.g., I could go manually
through the top 30 ones, say...

\subsection{Discovering morphological categories}
\label{sec:categories}

Focusing on German and Italian given massive morphosyntactic ambiguity
and impoverished morphology of English. Note that these are lexical
properties, probed in a model that has no explicit notion of word!

\paragraph{Word classes (nouns vs.~verbs)}

(Departures from general setup)

Procedure as described in the quip.

Baselines: autoencoder, word-based NLM embeddings, also LSTM vs
RNN. Outperforming autoencoder shows that model has learned categories
based on broader distributional evidence, not just typical strings
cueing nouns and verbs.

General accuracy results table for fixed N of training examples as in Table \ref{tab:pos-results}.

\begin{table}[t]
  \begin{center}
    \begin{tabular}{l|l|l}
      \multicolumn{1}{c}{}&\emph{German}&\emph{Italian}\\
      \hline
      LSTM&\ldots&\ldots\\
      RNN&\ldots&\ldots\\
      Autoencoder&\ldots&\ldots\\
      WordNLM&\ldots&\ldots\\
    \end{tabular}
  \end{center}
  \caption{\label{tab:pos-results} \ldots}
\end{table}

Report CNLM vs baseline comparison in function of training examples as in the Quip (possibly for a language only).

\paragraph{Number}

Does the hidden state of the CNLM store an abstract notion of
number. German nouns can be binned in different classes depending on
the morpheme or morphological process they use to form the plural. We
train a number classifier on a subset of these classes, test on the
others: if model generalizes correctly, it means that it knows about
number independently of its surface expression.

Specifics of data-set construction and classifier training. Pick only one setup, e.g., training on -n, -s, and -e.

Baselines/comparisons: LSTM vs RNN, autoencoder, word-based NLM.

Results could be summarized as in Table \ref{tab:number-results}.


\begin{table}[t]
  \begin{center}
    \begin{tabular}{l|l|l|l}
      \multicolumn{1}{c}{}&\emph{training plurals}&\emph{-r}&\emph{no-suffix}\\
      \hline
      LSTM&\ldots&\ldots\\
      RNN&\ldots&\ldots&\ldots\\
      Autoencoder&\ldots&\ldots&\ldots\\
      WordNLM&\ldots&\ldots&\ldots\\
    \end{tabular}
  \end{center}
  \caption{\label{tab:number-results} Figure of merit is accuracy for plural prediction.}
\end{table}


Control follow-up: singular nouns with plural ending (with similar table?).

\subsection{Capturing syntactic dependencies}
\label{sec:dependencies}

Despite not having pre-defined information about words and morphemes,
is the model able to capture non-adjacent syntactic dependencies? NB:
actually for a CNLM even \emph{``\textbf{la} bell\textbf{a}''} is long
distance! Constructions will be language-specific, so we discuss
German and Italian separately (not much in English).

As usual, specifics of training etc that depart from general setup.

\paragraph{German} We consider 4 constructions:
\begin{inparaenum}[i)]
\item article-noun gender agreement, possibly with material in the middle,
\item determiner-noun case concord, again with material in the middle,
\item preposition case sub-categorization, with material in the middle.
\end{inparaenum}

Discussion case-by-case, including how do we control for possible
n-gram effects and length. \textbf{It's not clear to me how the n-gram
  control model would work in each of these cases, can you explain in
  detail?}

How to present results: figures, with number of intervening words
(from 0 to 2 or 3) on x axis, accuracy on y axis. Multiple figures
when different genders or cases are tested. Within the figures, one
line per model: LSTM, RNN, n-gram, wordNLM.

\paragraph{Italian} We consider further constructions from Italian,
that confirm the results we got in German.
\begin{inparaenum}[i)]
\item article-noun gender agreement with material in the middle,
\item article-adjective gender agreement, with an adverb in the middle,
\item article-adjective  agreement, with an adverb in the middle.
\end{inparaenum}

Discussion case-by-case, including how we control for n-gram frequency
and length (NB: if we want to do article-noun gender agreement with
nothing in the middle, we must insert n-gram control).

Results: same figure format as above? Or table with a row for each
pattern and a column for each model?


\subsection{Lexical semantic similarity}
\label{sec:similarity}

In English, because that's where we have resources available.

Correlation with one or more word similarity sets.

Comparison to word-based NLM (rather than word2vec or such, which is
specifically tuned for semantics).



\section{Discussion}
\label{sec:discussion}

We probed the linguistic information induced by an LSTM language model
trained on unsegmented text at the character level. We found that the
model stores implicit knowledge about phonotactic constraints, word
units, major morphosyntactic classes, non-adjacent syntactic agreement
and subcategorization phenomena and even some degree of semantic
knowledge. While a more standard model pre-initialized with a word
vocabulary and reading tokenized input was in general superior on the
higher-level tasks, the performance of our agnostic model did not
generally lag much behind, suggesting that the word bias is helpful
but not fundamental. The performance of character-level RNN was
considerably less consistent than that of the equivalent LSTM, suggesting that
the ability of the latter to track information across longer time
steps is important to extract linguistic generalizations from the raw
input. Importantly, n-gram baselines only relying on adjacent string
statistics fail almost all tests, showing that the neural models
are tapping into somewhat deeper linguistic templates.

Our results are very preliminary in many ways. The tests we used are
generally simple \cite[we did not attempt, for example, to model
long-distance subject-verb agreement, a task that is challenging even
for word-based models:][]{Linzen:etal:2016}, and they only probe a
small subset of linguistic rules. Still, they do suggest that a large
corpus, combined with the very weak priors encoded in an LSTM, might
suffice to discover generalizations that appear to be of a genuine
linguistic nature.

One aspect that we find particularly intriguing is that, unlike the
standard word-based models, our CNLMs do not have a morpheme- or
word-based lexicon. Any information the network might acquire about
units larger than characters must be stored in its recurrent
weights. Given that nearly all contemporary linguistics recognizes a
central role to the lexicon \cite[see, e.g.,][for different
perspectives]{Sag:etal:2003,Goldberg:2005,Radford:2006,Bresnan:etal:2016,Jezek:2016},
in future work we would like to explore how lexical knowledge is
implicitly encoded in the distributed memory of our CNLMs.

One of our original motivations for not assuming word primitives is that a rigid word notion is problematic both cross-linguistically (cf.~polysynthetic and agglutinative languages) and when analyzing a single language (cf.~the common  view  that the lexicon hosts units at different levels of the linguistic hierarchy, from  morphemes to large syntactic constructions). Our brief analysis of the CNLM over- and undersegmentations suggested that it is indeed capable to flexibly store information about units at different levels. However, this topic  remained largely unexplored, and we plan to systematically tackle it in future work.



% remember to add to acks:

% sebastian r, hinrich s, alex c, german k, kristina g, piotr b

% add Karpathy's paper to ref

\bibliography{marco,michael}
\bibliographystyle{acl_natbib}



% \appendix

% \begin{table*}[t]
% 	\begin{tabular}{l|lll|lll|lllllll}
% 		&  \multicolumn{3}{c}{LSTM} & \multicolumn{3}{|c|}{RNN} & \multicolumn{3}{c}{WordNLM} \\
% 		       &  En.     &  Ge.    & It.    & En.    &    Ge.   &  It.     &  En.     &   Ge.   &    It. \\  \hline
% 	Batch Size     &  128   &  512  & 128  & 256  & 256    &  256   &  128   &   128 &  128   \\              
% 	Embedding Size &  200   &  100  & 200  & 200  & 50     &  50    &  1024  &   200 &  200   \\             
% 	Dimension      &  1024  &  1024 & 1024 & 2048 & 2048   &  2048  &  1024  &  1024 &  1024  \\  
% 	Layers         &  3     &  2    & 2    & 2    & 2      &  2     &  2     &  2    &  2     \\   
% 	Learning Rate  &  3.6   &  2.0  & 3.2  & 0.01 & 0.1    &  0.1   &  1.1   &  0.9  &  1.2   \\ 
% 	Decay          &  0.95  &  1.0  & 0.98 & 0.9  & 0.95   &  0.95  &  1.0   &  1.0  &  0.98  \\
% 	BPTT Length    &  80    &  50   & 80   & 50   & 30     &  30    &  50    &  50   &  50    \\
% 	Hidden Dropout &  0.01  &  0.0  & 0.0  & 0.05 & 0.0    &  0.0   &  0.15  &  0.15 &  0.05  \\   
% 	Embedding Dropout  & 0.0& 0.01  & 0.0  & 0.01 & 0.0    &  0.0   &  0.0   &  0.1  &  0.0   \\   
% 	Input Dropout  & 0.001 &  0.0   & 0.0  & 0.001& 0.01   &  0.01  &  0.01  &  0.001&  0.01  \\ 
%         Nonlinearity   &   --  & --     & --   & ReLu & tanh   &  tanh  &   --   &  --   &  --    \\                   
% \end{tabular}
% 	\caption{Hyperparameters identified \textbf{probably we cannot put this into the submission within the 10 page limit?}}
% \end{table*}



\end{document}


