
\subsubsection{Italian} \textbf{Also here, add data-set sizes.}
\textbf{For the examples, you know the linguex packet, right?}
\textbf{Given that you itemize by gender, can you add a baseline based
  on picking the most frequent variant of the noun/adjective?} We
focus on a subset of Italian morphology where gender and number are
explicitly and systematically encoded while allowing for tightly
controlled comparison of same-length strings. We are able to extract
enough stimuli that never occur in the training corpus, so that an
n-gram model would be at chance level.

\paragraph{Article-noun gender agreement}
%(1) eadj-aonoun:

Similar to German, Italian articles agree with the noun in gender; however, Italian has a relatively extended paradigm of feminine and masculine nouns differing only in the final vowel vowel (-\emph{o} and -\emph{a}, respectively). We construct stimuli of the form:
\begin{enumerate}[label={(\arabic*)}]
	\item 
		\begin{tabular}[t]{lllllll}
	a. & \{\underline{il}, la\} & congeniale & candidato \\
   &  the & congenial & candidate (m.) \\
	& \multicolumn{4}{l}{`The congenial male candidate.'} \\
	b. & \{il, \underline{la}\} & congeniale & candidata \\
    &the & congenial & candidate (f.) \\
	& \multicolumn{4}{l}{`The congenial female candidate.'} \\
\end{tabular}
\end{enumerate}

The intervening adjective, ending in -\emph{e}, does not reveal the
noun's gender, increasing the distance across which gender information
has to be transported. We constructed the stimuli out of single words
appearing at least 100 times in the training corpus. We required
moreover that the \emph{-a} and \emph{-o} forms of a noun are
reasonably balanced in frequency (neither form is twice more frequent
than the other), or both rather frequent (appear at least 500
times). As the prenominal adjectives are somewhat marked, we only
considered -\emph{e} adjectives that occur prenominally with at least
10 distinct nouns in the training corpus. Here and below, stimuli
where checked for strong semantic anomalies.

Results are shown in the first line of
Table~\ref{tab:ital-agr-results}.  The word LSTM shows the strongest
performance, closely followed by the LSTM CNLM.  Even the RNN CNLM
performs strongly above chance, but again lags behind the LSTM.

\paragraph{Article-adjective gender agreement}
We next consider agreement between articles and adjectives with an intervening an adverb:
\begin{enumerate}[label={(\arabic*)}]
	\item 
		\begin{tabular}[t]{lllllll}
	a. & il & meno & \{ \underline{alieno}, aliena \} \\
   &  the (m.)& less & alien  \\
	b. & la & meno & \{ alieno, \underline{aliena} \} \\
    &the (f.)& less & alien one \\
\end{tabular}
\end{enumerate}
where we used the adverbs \emph{pi{\`u}} `more', \emph{meno} `less',
\emph{tanto} `so much'. We considered only adjectives that occurred 1K
times in the training corpus (as \emph{-a}/\emph{-o} adjectives are
the most common class). We excluded all cases in which the
adverb-adjective combination occurred in the training corpus. %
% /checkpoint/mbaroni/char-rnn-exchange/candidate_adv_aoadj_testset.txt
Results are shown in the second line of Table~\ref{tab:ital-agr-results}; all three models perform almost perfectly.

\begin{table}[t]
  \begin{center}
    \begin{tabular}{l|ll|ll|ll}
	    & \multicolumn{4}{c|}{CNLM} & \multicolumn{2}{c}{\multirow{2}{*}{WordNLM}}\\
	    &\multicolumn{2}{c|}{\emph{LSTM}}&\multicolumn{2}{c|}{\emph{RNN}} &\\ \hline
% eadj-aonoun
	    Noun Gender & 97&90  & 84&73 & 99&96 \\
%      adv-aoadj
	    Adj.~Gender & 99&100 & 100&97 & 98&100 \\
% adv-aeadj
	    Adj.~Number & 99&99 & 99&70 & 100&100 \\
    \end{tabular}
  \end{center}
	\caption{\label{tab:ital-agr-results} Italian agreement results. For each model and test, we report percentage accuracy on two stimulus classes (masculine/feminine for gender, singular/plural for number).}
\end{table}

\paragraph{Article-adjective number agreement}
Finally, we constructed a version of the last test that probed number agreement; number marking is similarly very systematic in Italian, and for feminine forms (illustrated below) it's possible to compare same-length phrases:
\begin{enumerate}[label={(\arabic*)}]
	\item 
\begin{tabular}[t]{lllllll}
	a. & la & meno & \{ \underline{aliena}, aliene \} \\
   &  the (s.)& less & alien one(s)  \\
	b. & le & meno & \{ aliena, \underline{aliene} \} \\
    &the (p.)& less & alien one(s) \\
\end{tabular}
\end{enumerate}
% /checkpoint/mbaroni/char-rnn-exchange/candidate_adv_aeadj_testset.txt
Selection of stimuli was as above, but we used a 500-occurrences
threshold, as feminine plurals are less common. %  Further, we manually
% removed adjectives that did not combine well semantically with the
% adverbs under consideration (\emph{pi{\`u}, meno, tanto}).
Results are shown in the third line in Table~\ref{tab:ital-agr-results}; the LSTMs perform almost perfectly, while the RNN still performs strongly above chance.

