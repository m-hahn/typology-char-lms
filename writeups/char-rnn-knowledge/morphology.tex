\subsection{Discovering morphological categories}
\label{sec:categories}

Words belong to part-of-speech categories, such as nouns and
verbs. Moreover, they typically carry inflectional features such as
number. We start by probing whether CNLMs capture such
properties. We use here the popular method of ``diagnostic
classifiers'' \cite{Hupkes:etal:2017}. That is, we treat the hidden
activations produced by a CNLM whose weights were fixed after language
model training as input features for a shallow (logistic) classifier
of the property of interest (e.g., plural vs.~singular). If the
classifier is successful, this means that the representations provided
by the model are encoding the relevant information.  The classifier is
deliberately shallow and trained on a small set of examples, as we
want to test whether the properties of interests are robustly encoded
in the representations produced by the CNLMs, and amenable to a simple
linear readout \cite{Fusi:etal:2016}. In our case, we want to probe
word-level properties in models trained at the character level. To do
this, we let the model read each target word character-by-character,
and we treat the state of its hidden layer after processing the last
character in the word as the model's implicit representation of the
word, on which we train the diagnostic classifier. %
% Besides being sensitive, to some extent, to word boundaries, does
% the CNLM also store linguistic properties of words, such as their part
% of speech and number?
The experiments focus on German and Italian, as it's harder to design
reliable test sets for the impoverished English morphological system.

\paragraph{Word classes (nouns vs.~verbs)}

For both German and Italian, we sampled 500 verbs and 500 nouns from
the Wikipedia training sets, requiring that they are unambiguously
tagged in the corpus by TreeTagger. Verbal and nominal forms are often
cued by suffixes. We removed this confound by selecting examples with
the same ending across the two categories (\emph{-en} in German:
\emph{Westen} `west',\footnote{German nouns are capitalized; this cue
  is unavailable to the CNLM as we lower-case the input.} %
\emph{stehen} `to stand'; and \emph{-re} in Italian: \emph{autore}
`author', \emph{dire} `to say'). We randomly selected 20 training examples
(10 nouns and 10 verbs), and tested on the remaining items.  We
repeated the experiment 100 times to account for random train-test
split variation.
%  We recorded the final
% hidden state of a pre-trained CNLM after reading a word, without
% context, and trained a logistic noun-verb classifier on these
% representations.

While we controlled for suffixes as described above, it could still be
the case that other substrings reliably cue verbs or nouns. We thus
considered a baseline trained on word-internal information only,
namely a character-level LSTM autoencoder trained on the Wikipedia datasets to reconstruct words 
in isolation.\footnote{For each language, autoencoder hyperparameters were chosen using random search, as for the language models; details are in supplementary material to be made available upon publication. For both German and Italian models, the following parameters were chosen: 2 layers, 100 embedding dimensions, 1024 hidden dimensions.}
The hidden state of the LSTM autoencoder should capture
discriminating orthographic features, but, by design, will have
no access to broader contexts.  We further considered word embeddings
from the output layer of the WordNLM. Unlike CNLMs, the WordNLM cannot
make educated guesses about words that are not in its training
vocabulary. These OOV words are by construction less frequent, and
thus likely to be in general more difficult. To get a sense of both
``best-case-scenario'' and more realistic WordNLM performance, we
report its accuracy both excluding and including OOV items
(WordNLM$_{\textit{subs.}}$ and WordNLM in Table
\ref{tab:pos-results}, respectively). In the latter case, we let the
model make a random guess for OOV items.  The percentage of OOV items
over the entire dataset, balanced for nouns and verbs, was 92.3\% for
German and 69.4\% for Italian.

Results are in Table~\ref{tab:pos-results}.  All language models
outperform the autoencoders, showing that they learned categories
based on broader distributional evidence, not just typical strings
cuing nouns and verbs. Moreover, the LSTM CNLM outperforms the RNN,
probably because it can track broader contexts. Not surprisingly, the
word-based model fares better on in-vocabulary words, but the gap,
especially in Italian, is rather narrow, and there is a strong
negative impact of OOV
words (as expected, given that WordNLM is at random on them). % Figure~\ref{fig:pos-induction} shows how German performance evolves as the training set grows from 2 to 100 examples (Italian results are qualitatively identical). The CNLMs already distinguish the categories well with small training sets, while the autoencoder does not catch up even with 100 training examples per category.

\begin{table}[t]
%  \begin{small}
\footnotesize
    \begin{center}
      \begin{tabular}{l|l|l}
        &\emph{German}&\emph{Italian}\\
        \hline
	      Random & 50.0 & 50.0 \\
        Autoencoder & 65.1 ($\pm$ 0.22) & 82.8 ($\pm$ 0.26) \\
	      \hline
        LSTM & 89.0 ($\pm$ 0.14) & 95.0 ($\pm$ 0.10) \\
        RNN & 82.0 ($\pm$ 0.64) & 91.9 ($\pm$ 0.24) \\
	      WordNLM & 53.5 ($\pm$ 0.18)  & 62.5 ($\pm$ 0.26) \\ \hline
	      WordNLM$_{\textit{subs.}}$ & 97.4 ($\pm$ 0.05) & 96.0 ($\pm$ 0.06) \\
      \end{tabular}
    \end{center}
 % \end{small}
	\caption{\label{tab:pos-results} Accuracy of diagnostic classifier on predicting word class, with standard errors across 100 random train-test splits.~`\emph{subs.}'~marks in-vocabulary subset evaluation, not comparable to the other results.} % Random accuracy is 50\%.} % (20 training examples)
\end{table}


% \begin{figure}
% \includegraphics[width=0.48\textwidth]{figures/german_pos_nouns_verbs.pdf}
% 	\caption{Word class accuracy as a function of training examples (German). }\label{fig:pos-induction}
% 	% \textbf{Please rename LM LSTM, Baseline Autoencoder and Words WordNLM}
% \end{figure}




\paragraph{Number}
We turn next to number, a more granular morphological feature. We
study German, as it possesses a rich system of nominal classes forming
plural through different morphological processes. We train a diagnostic number
classifier on a subset of these classes, and test on the others, in order to probe the abstract number generalization capabilities of the tested models. If a
model generalizes correctly, it means that the CNLM is sensitive to number
as an abstract feature, independently of its surface expression.

We extracted plural nouns from the Wiktionary and the German UD
treebank \cite{mcdonald2013universal,brants2002tiger}.  We
selected % de2006generating,
nouns with plurals in -\emph{n}, -\emph{s}, or -\emph{e} to train the
classifier (e.g., \emph{Geschichte(n)} `story(-ies)', \emph{Radio(s)}
`radio(s)', \emph{Pferd(e)} `horse(s)', respectively). We tested on
plurals formed with -\emph{r} (e.g., \emph{Lieder} for singular
\emph{Lied} `song'), or through vowel change (\emph{Umlaut}, e.g.,
\emph{{\"A}pfel} from singular \emph{Apfel} `apple'). Certain nouns
form plurals through concurrent suffixing and Umlaut. We grouped
these together with nouns using the same suffix, reserving the Umlaut
group for nouns \emph{only} undergoing vowel change (e.g.,
\emph{Saft/S\"afte} `juice(s)' would be an instance of -\emph{e}
suffixation). The diagnostic classifier was trained on 15 singulars and plurals
randomly selected from each training class.  As plural suffixes make
words longer, we sampled singulars and
plurals %used rejection sampling \textbf{(cite something?)}
from a single distribution over lengths, to ensure that their lengths
were approximately matched. Moreover, since in uncontrolled samples
from our training classes a final \emph{-e-} vowel would constitute a
strong surface cue to plurality, we balanced the distribution of this
property across singulars and plurals in the samples. For the test
set, we selected all plurals in -\emph{r} (127) or Umlaut (38), with
their respective singulars. %\textbf{(how many?)}.
We also used all remaining plurals ending in -\emph{n} (1467),
-\emph{s} (98) and -\emph{e} (832) as in-domain test data.
%\textbf{Need information on the test set for the training classes.}
To control for the impact of training sample selection, we report
accuracies averaged over 200 random train-test splits and
standard errors over these splits. %   We extract word representations as
% above, and we compare the same models.
For WordNLM OOV, there were 45.0\% OOVs in the training classes,
49.1\% among the -\emph{r} forms, and 52.1\% for Umlaut.
%\textbf{MICHAEL: I didn't
%  realize you selected matched singular/plural pairs. Is this the
%  case? Anyway, wouldn't the more informative statistics here be the one
%  about missing items independently of whether they are plural or
%  singular? I think that's what matters, or else the high performance
%  in the full data-set would be miraculous...}

Results are in Table \ref{tab:number-results-e}. The classifier based
on word embeddings is the most successful. It outperforms in most cases
the best CNLM even in the more cogent OOV-inclusive evaluation. This
confirms the common observation that word embeddings reliably encode
number \cite{Mikolov:etal:2013a}. Again, the LSTM-based CNLM is better
than the RNN, but both significantly outperform the autoencoder. The
latter is near-random on new class prediction, confirming that we
properly controlled for orthographic confounds.

We observe a considerable drop in the LSTM CNLM performance between
generalization to -\emph{r} and Umlaut. On the one hand, the fact that
performance is still clearly above chance (and autoencoder) in the
latter condition shows that the LSTM CNLM has a somewhat abstract
notion of number not tied to specific orthographic exponents. On the
other, the -\emph{r} vs.~Umlaut difference suggests that the
generalization is not completely abstract, as it works more reliably
when the target is a new suffixation pattern, albeit one that is
distinct from those seen in training, than when
it is a purely non-concatenative process.

% We found that training classes tend to have an unstressed final -e- in
% the plural only, either as a plural ending (\emph{Abend} `evening' $\rightarrow$ \emph{Abend-e}) or as an epenthetic vowel
% intervening before a consonantal plural suffix (\emph{Autor} `author' $\rightarrow$ \emph{Autor-e-n}).
% However, all singulars
% that use only Umlaut to form their plurals have an unstressed final -e-
% in both singular and plural (e.g., \emph{Apfel} `apple' $\rightarrow$ \emph{{\"A}pfel}). 
% If the diagnostic classifier is sensitive to final unstressed -e- as a cue for plurals,
% it is expected that
% it should classify singulars and plurals as plurals.
% Indeed, we found that the CNLM classifiers mistook
% most Umlaut singulars for plurals.  On one randomly chosen split,
% 100\% of both singulars and plurals were classified as plurals.  
% No such problem occurs for -\emph{r} plurals, in contrast, as they
% follow the pattern of the training classes (e.g., \emph{Lied} `song' $\rightarrow$ \emph{Lied-e-r}).

% \textbf{Marco: Regarding your counterexamples (Kuss, Wurm), they fall under -e/-r plurals in the experiment. I clarified this at the beginning of this section.}

%By inspecting the results, we found that the CNLM classifiers mistook
%most Umlaut singulars for plurals.  On one randomly chosen split,
%100\% of both singulars and plurals were classified as plurals.  We
%observed that singulars using only Umlaut to mark plural have the
%commonality that their last vowel is an (unstressed) -\emph{e}-.  This
%vowel -\emph{e}- is also common in -\emph{r} plurals, many of which
%add a vowel -\emph{e}- between the consonant-final stem and the ending
%-\emph{r}.  Based on this observation, we hypothesized that the
%CNLM-based classifier detects the presence of -\emph{e}- towards the
%end of the word, which would explain why it does well on -\emph{r}
%plurals and at chance of Umlaut plurals. \textbf{MICHAEL: I think the
%  previous paragraph needs to be clarified in a few ways. First, it is
%  NOT always the case that Umlaut-triggering singulard have an
%  unstressed e as last vowel, Kuss, Wurm, etc. Do you mean that this
%  is the case in your data set? Then, this should be clarified (and
%  explained). Second, I think the argument would be clearer by turning
%  it around: Our training classes tend to have an unstressed final -e-
%  in the plural only. Many singulars in Umlaut have an unstressed
%  final -e-, hence the mishap. No problem with -r, as it follows the
%  pattern of the training classes. Finally, give examples!}


\begin{table}[t]
	\footnotesize
  \begin{center}
    \begin{tabular}{@{\hspace{0.3em}}l@{\hspace{0.42em}}|@{\hspace{0.42em}}l@{\hspace{0.45em}}|@{\hspace{0.45em}}l@{\hspace{0.65em}}l@{\hspace{0.15em}}}
      &train classes&\multicolumn{2}{c}{test classes}\\
      &\emph{-n/-s/-e}&\multicolumn{1}{c}{\emph{-r}}&\multicolumn{1}{c}{\emph{Umlaut}}\\      \hline
	    Random & 50.0 & 50.0 & 50.0 \\
	    Autoencoder & 61.4 ($\pm$ 0.9)  & 50.7 ($\pm$ 0.8)  & 51.9 ($\pm$ 0.4)  \\            \hline
	    LSTM & 71.5 ($\pm$ 0.8)  & 78.8 ($\pm$ 0.6)  & 60.8 ($\pm$ 0.6)  \\
	    RNN & 65.4 ($\pm$ 0.9)  & 59.8 ($\pm$ 1.0)  & 56.7 ($\pm$ 0.7)  \\
	    WordNLM  & 77.3 ($\pm$ 0.7)  & 77.1 ($\pm$ 0.5)  & 74.2 ($\pm$ 0.6)  \\ \hline
	    WordNLM$_{\textit{subs.}}$ & 97.1 ($\pm$ 0.3)  & 90.7 ($\pm$ 0.1)  & 97.5 ($\pm$ 0.1)  \\
    \end{tabular}
  \end{center}
  \caption{\label{tab:number-results-e} German number classification
	accuracy , with standard errors computed from 200 random train-test % when controlling for -\emph{e}-
	splits.~`\emph{subs.}'~marks in-vocabulary subset evaluation, not comparable to the other results.} % Random accuracy is 50\%.}
\end{table}


% PREVIOUS VERSION FROM HERE
% We turn next to number, a more granular morphological feature. We
% study German, as it possesses a rich system of nominal classes forming
% plural through different morphological processes. We train a diagnostic number
% classifier on a subset of these classes, and test on the others. If a
% model generalizes correctly, it means that the CNLM is sensitive to number
% as an abstract feature, independently of its surface expression.

% We extracted plural nouns from the Wiktionary and the German UD
% treebank \cite{mcdonald2013universal,brants2002tiger}.  We selected % de2006generating,
% nouns with plurals in -\emph{n}, -\emph{s}, or -\emph{e} to train the classifier (e.g., \emph{Geschichte-n} `stories'), and tested on plurals formed with
% -\emph{r} or through vowel change (\emph{Umlaut}, e.g., \emph{T{\"o}chter} for singular \emph{Tochter} `daughter'). \textbf{MICHAEL: maybe better to put also examples with -r, or at least not an Umlaut example ending in -r?}

% %\textbf{Add one
% %  example from training, one from testing.}

% For the training set, we randomly selected 15 singulars and plurals
% from each training class.  As plural suffixes make words longer, we
% sampled singulars and
% plurals %used rejection sampling \textbf{(cite something?)}
% from a single distribution over lengths, to ensure that their lengths
% were approximately matched.  For the test set, we selected all plurals
% in -\emph{r} (127) or Umlaut (38), with their respective
% singulars. %\textbf{(how many?)}.
% We also used all remaining plurals ending in -\emph{n} (1467), -\emph{s} (98) and -\emph{e} (832) as in-domain test data.
% %\textbf{Need information on the test set for the training classes.}
% To control for the impact of training sample selection, we
% report accuracies averaged over 200 random train-test splits and report standard errors over these splits.  We extract word
% representations as above, and we compare the same models. \textbf{MICHAEL: proportions of OOVs here.} %
% %, excluding OOVs when testing the latter.
% Results are summarized in Table \ref{tab:number-results}.

% \begin{table}[t]
% 	\footnotesize
%   \begin{center}
%     \begin{tabular}{@{\hspace{0.3em}}l@{\hspace{0.42em}}|@{\hspace{0.42em}}c@{\hspace{0.45em}}|@{\hspace{0.45em}}l@{\hspace{0.65em}}l@{\hspace{0.15em}}}
%       &train classes&\multicolumn{2}{c}{test classes}\\
%       &\emph{-n/-s/-e}&\multicolumn{1}{c}{\emph{-r}}&\multicolumn{1}{c}{\emph{Umlaut}}\\      \hline
% 	    LSTM & 73.5 ($\pm$ 0.7)  & 89.4 ($\pm$ 0.3)  & 53.7 ($\pm$ 0.3)  \\
% 	    RNN & 67.0 ($\pm$ 0.9)  & 83.5 ($\pm$ 0.5)  & 53.1 ($\pm$ 0.4)  \\
% 	    Autoencoder & 63.6 ($\pm$ 0.9)  & 78.1 ($\pm$ 0.5)  & 52.8 ($\pm$ 0.2)  \\
% 	    WordNLM$_{\textit{subs.}}$  & 97.4 ($\pm$ 0.3)  & 90.5 ($\pm$ 0.1)  & 97.3 ($\pm$ 0.1)  \\ 
% 	    WordNLM  & 78.2 ($\pm$ 0.7)  & 77.8 ($\pm$ 0.4)  & 75.9 ($\pm$ 0.5)  \\ 
%     \end{tabular}
%   \end{center}
%   \caption{\label{tab:number-results} German number classification
%     accuracy, with standard errors computed from 200 random train-test
%     splits.  `\emph{subs.}' marks in-vocabulary subset evaluation, not
%     comparable to the other results.}
% \end{table}

% % python char-lm-ud-stationary-separate-bidir-with-spaces-probe-baseline-prediction-wiki-plurals-2-tests-redone-wikisource.py --language german --batchSize 128 --char_embedding_size 100 --hidden_dim 1024 --layer_num 2 --weight_dropout_in 0.1 --weight_dropout_hidden 0.35 --char_dropout_prob 0.0 --char_noise_prob 0.01 --learning_rate 0.2 --load-from wiki-autoencoder

% % python char-lm-ud-stationary-separate-bidir-with-spaces-probe-baseline-prediction-wiki-plurals-2-tests-words-wikisource.py  --language german --batchSize 128 --char_embedding_size 200 --hidden_dim 1024 --layer_num 2 --weight_dropout_in 0.1 --weight_dropout_hidden 0.35 --char_dropout_prob 0.0 --char_noise_prob 0.01 --learning_rate 0.2 --load-from wiki-german-nospaces-bptt-words-966024846



% The classifier based on word embeddings is the most successful,
% outperforming in most cases the best CNLM even in the more cogent
% OOV-inclusive evaluation. This confirms the common observation that
% word embeddings reliably encode number \cite{Mikolov:etal:2013a}. The
% CNLM results are mixed. Again the LSTM variant outperforms the
% RNN, but neither has problems detecting number for new words from
% the training paradigms. They are also able to detect number in a new
% paradigm (\emph{-r} plurals), seemingly suggesting that they are, to some extent,
% inducing an abstract notion of number that is not tied to specific
% orthographic exponents. In these experiments, the CNLMs significantly
% outperform the autoencoder. %, and the LSTM even does better than the
% %word model in \emph{-r}-class generalization.

% In contrast, the CNLMs fail to generalize to Umlaut plurals, where
% they are virtually at chance level, similar to the autoencoder. %and \emph{worse} than the
% %autoencoder.
% % The most obvious difference between this type and
% % \emph{-r} pluralization is that the latter is an affixation process,
% % whereas Umlaut pluralization changes the stressed vowel of the
% % stem.
% By inspecting the results, we found that the CNLM classifiers mistook
% most Umlaut singulars for plurals.  On one randomly chosen split,
% 100\% of both singulars and plurals were classified as plurals.  We
% observed that singulars forming their plural with Umlaut have the
% commonality that their last vowel is an (unstressed) -\emph{e}-.  This
% vowel -\emph{e}- is also common in -\emph{r} plurals, many of which
% add a vowel -\emph{e}- between the consonant-final stem and the ending
% -\emph{r}.  Based on this observation, we hypothesized that the
% CNLM-based classifier detects the presence of -\emph{e}- towards the
% end of the word, which would explain why it does well on -\emph{r}
% plurals and at chance of Umlaut plurals. \textbf{MICHAEL: I think the
%   previous paragraph needs to be clarified in a few ways. First, it is
%   NOT always the case that Umlaut-triggering singulard have an
%   unstressed e as last vowel, Kuss, Wurm, etc. Do you mean that this
%   is the case in your data set? Then, this should be clarified (and
%   explained). Second, I think the argument would be clearer by turning
%   it around: Our training classes tend to have an unstressed final -e-
%   in the plural only. Many singulars in Umlaut have an unstressed
%   final -e-, hence the mishap. No problem with -r, as it follows the
%   pattern of the training classes. Finally, give examples!}

% To test this hypothesis, we created a na\"ive ``classifier'' that
% labels a word as plural if one of the last two characters is \emph{e}.
% On the full dataset, this classification method achieves 73.7 \%
% accuracy on the train classes, 95.1\% on -\emph{r} plurals, and 51.0\%
% on Umlaut plurals.  This accuracy is close to that of the LSTM CNLM on
% in-domain and Umlaut, and even outperforms it on -\emph{r}.  Across
% the 200 runs, average agreement between classification decisions made
% with this simple heuristic and the CNLM-based ones was 75.0\%
% ($\pm$0.8) on the in-domain classes, 89.5 ($\pm$ 0.3) on the
% -\emph{r}, and 95.4 ($\pm$0.4) on Umlaut.  We conclude that, in the
% experiment described above, the CNLM classifier was strongly
% influenced by a very superficial cue (presence of -\emph{e}- near the
% word end) that correlates with plural-hood reasonably well, but does
% not generalize well across morphological classes of plural formation.

% % python char-lm-ud-stationary-separate-bidir-with-spaces-probe-baseline-prediction-wiki-plurals-2-tests-redone-wikisource-investigateE.py --language german --batchSize 128 --char_embedding_size 100 --hidden_dim 1024 --layer_num 2 --weight_dropout_in 0.1 --weight_dropout_hidden 0.35 --char_dropout_prob 0.0 --char_noise_prob 0.01 --learning_rate 0.2 --load-from wiki-german-nospaces-bptt-910515909

% Informed by this result, we repeated the experiment with controlled
% training sets where incidence of -\emph{e}- was balanced across
% singulars and plurals.  To balance for the plurals, we made sure that
% 1/4 of those ending in -r or -s had an -\emph{e}- as their
% second-to-last character, so that the total fraction of training
% plurals with -\emph{e}- in the last two characters was 1/2.  For
% singulars, we balanced the incidence of -\emph{e}- for each of the
% classes.  Resulting accuracies are shown in
% Table~\ref{tab:number-results-e}.

% Performance of the diagnostic classifier for the LSTM CNLM is weaker than before for -\emph{r} plurals, and stronger than before for umlaut plurals.
% That is, performance is weaker where the superficial cue used before (-\emph{e}-) was helpful (-\emph{r} plurals), and stronger where the cue is misleading (umlaut plurals).
% %This is expected, as we had found that the performance in the first experiment could be explained mostly by presence of -\emph{e}-.
% The autoencoder performs essentially at chance on held-out classes, showing that its previous modest success can be explained by the -\emph{e}- cue.
% This shows that the CNLM encodes something more abstract than the autoencoder, but, overall, there is only modest evidence that the CNLM encodes any abstract notion of nominal plural in its hidden state.



% \begin{table}[t]
% 	\footnotesize
%   \begin{center}
%     \begin{tabular}{@{\hspace{0.3em}}l@{\hspace{0.42em}}|@{\hspace{0.42em}}c@{\hspace{0.45em}}|@{\hspace{0.45em}}l@{\hspace{0.65em}}l@{\hspace{0.15em}}}
%       &train classes&\multicolumn{2}{c}{test classes}\\
%       &\emph{-n/-s/-e}&\multicolumn{1}{c}{\emph{-r}}&\multicolumn{1}{c}{\emph{Umlaut}}\\      \hline
% 	    LSTM & 71.5 ($\pm$ 0.8)  & 78.8 ($\pm$ 0.6)  & 60.8 ($\pm$ 0.6)  \\
% 	    RNN & 65.4 ($\pm$ 0.9)  & 59.8 ($\pm$ 1.0)  & 56.7 ($\pm$ 0.7)  \\
% 	    Autoencoder & 61.4 ($\pm$ 0.9)  & 50.7 ($\pm$ 0.8)  & 51.9 ($\pm$ 0.4)  \\
% 	    WordNLM$_{\textit{subs.}}$ & 97.1 ($\pm$ 0.3)  & 90.7 ($\pm$ 0.1)  & 97.5 ($\pm$ 0.1)  \\
% 	    WordNLM  & 77.3 ($\pm$ 0.7)  & 77.1 ($\pm$ 0.5)  & 74.2 ($\pm$ 0.6)  \\
%     \end{tabular}
%   \end{center}
%   \caption{\label{tab:number-results-e} German number classification
% 	accuracy when controlling for -\emph{e}-, with standard errors computed from 200 random train-test
%     splits.  `\emph{subs.}' marks in-vocabulary subset evaluation, not comparable to the other results.}
% \end{table}

% PREVIOUS VERSION TO HERE

% python char-lm-ud-stationary-separate-bidir-with-spaces-probe-baseline-prediction-wiki-plurals-2-tests-words-wikisource-balanceE.py  --language german --batchSize 128 --char_embedding_size 200 --hidden_dim 1024 --layer_num 2 --weight_dropout_in 0.1 --weight_dropout_hidden 0.35 --char_dropout_prob 0.0 --char_noise_prob 0.01 --learning_rate 0.2 --load-from wiki-german-nospaces-bptt-words-966024846

% python char-lm-ud-stationary-separate-bidir-with-spaces-probe-baseline-prediction-wiki-plurals-2-tests-redone-wikisource-balanceE.py --language german --batchSize 128 --char_embedding_size 100 --hidden_dim 1024 --layer_num 2 --weight_dropout_in 0.1 --weight_dropout_hidden 0.35 --char_dropout_prob 0.0 --char_noise_prob 0.01 --learning_rate 0.2 --load-from wiki-german-nospaces-bptt-910515909

% python char-lm-ud-stationary-separate-bidir-with-spaces-probe-baseline-prediction-wiki-plurals-2-tests-redone-wikisource-balanceE-RNN.py --batchSize 256 --char_dropout_prob 0.01 --char_embedding_size 50 --char_noise_prob 0.0 --hidden_dim 2048 --language german --layer_num 2 --learning_rate 0.1 --lr_decay 0.95 --nonlinearity tanh --load-from wiki-german-nospaces-bptt-rnn-237671415 --sequence_length 30 --verbose True --weight_dropout_hidden 0.0 --weight_dropout_in 0.0

% python char-lm-ud-stationary-separate-bidir-with-spaces-probe-baseline-prediction-wiki-plurals-2-tests-words-wikisource-balanceE-withOOV.py  --language german --batchSize 128 --char_embedding_size 200 --hidden_dim 1024 --layer_num 2 --weight_dropout_in 0.1 --weight_dropout_hidden 0.35 --char_dropout_prob 0.0 --char_noise_prob 0.01 --learning_rate 0.2 --load-from wiki-german-nospaces-bptt-words-966024846






 













%Thus, a cautious interpretation is that the CNLMs developed a
%representation of plural that is sufficiently abstract to generalize
%to different affixation processes than the ones seen in training,
%but. Note that, as we control for length, we are confident that the
%network is not simply relying on this superficial feature to detect
%plurals (which would explain why affixation cases are
%easier). Similarly, the network can't simply rely on consonants cuing
%plurals, both because many singular forms also end with consonants, and
%because one of the training classes had \emph{-e} as plural marker.
%\textbf{We need some kind of story here!!!}
%% Evidently, CNLM number encoding is not abstract enough to
%% generalize across very different surface morphological processes
%% (adding a suffix vs.~changing the root vowel).
%
%Interestingly, many of the Umlaut words end in \emph{-n} and \emph{-r}, and these singulars are mostly misclassified as plurals. 
%We hypothesized that the CNLMs-based classifiers classify number based on the orthographic endings, not based on abstract number representations.
%To test this, we extracted from Wiktionary singular nouns ending in \emph{-n, -s, -e, -r}, excluding nouns that have identical singular and plural forms.
%We also extracted nouns ending in one of the two most common final characters of singular nouns in the dataset that can \emph{not} act as the last character of a plural, which are \emph{g} and \emph{t}.
%We then ran the number classifier on these nouns, again with 200 random choices of the training set.

%Resulting accuracies are shown in Table~\ref{tab:number-foils-results}.
%Classifiers based on CNLM or autoencoder representations show relatively poor accuracies on the singulars ending in characters typical of plurals (\emph{-n, -s, -e, -r}), i.e., the diagnostic classifiers mistake many singulars for plurals based on their ending.
%Conversely, they mostly recognize singulars when the final consonant makes this unambiguous.
%This suggests that -- to the extent accessible to a shallow diagnostic classifier -- the CNLM encodes not an abstract number feature, but orthographic features typical of plurals.
%
%\begin{table}[t]
%	\footnotesize
%  \begin{center}
%    \begin{tabular}{@{\hspace{0.3em}}l@{\hspace{0.42em}}|@{\hspace{0.42em}}c@{\hspace{0.45em}}@{\hspace{0.45em}}l@{\hspace{0.65em}}|l@{\hspace{0.15em}}}
%	    &\emph{-n}/\emph{-s}/\emph{-e}&\emph{-r}&\emph{-g}/\emph{-t}  \\ \hline
%	    LSTM& 65.2 ($\pm$ 0.7)  & 55.1 ($\pm$ 0.2) & 83.2 ($\pm$ 0.4) \\
%	    RNN& 61.3 ($\pm$ 0.2) &  51.6 ($\pm$ 0.2) & 75.5 ($\pm$ 0.1) \\
%	    Autoencoder& 65.7 ($\pm$ 0.3) & 50.0 ($\pm$ 0.2) & 72.5 ($\pm$ 0.2) \\
%	    WordNLM$_{\textit{subs.}}$& 93.4 ($\pm$ 0.1) & 95.2 ($\pm$ 0.3) & 93.3 ($\pm$ 0.2) \\
%	    WordNLM & 86.3 ($\pm$ 0.1) & 88.8 ($\pm$ 0.2) & 88.4 ($\pm$ 0.3) \\
%    \end{tabular}
%  \end{center}
%	\caption{\label{tab:number-foils-results} Accuracy on singulars ending in frequent characters that cue plurals (n, s, e, r), or are incompatible with plurals (g, t). }
%\end{table}
%


