\subsection{Discovering morphological categories}
\label{sec:categories}

A fundamental characteristic of words is that they belong to
part-of-speech categories such as nouns and verbs. Moreover, in
synthetic languages such as those we study, they carry inflectional
features such as number. We start by probing whether the
representations learned by our CNLM contains information about such
properties. We use here the popular model of ``diagnostic
classifiers'' \cite{Hupkes:etal:2017}. That is, we use the fixed,
language-modeling-trained representations produced by the CNLM to
train a shallow (logistic) classifier to distinguish the property of
interest (e.g., plural vs.~singular). If the classifier is successful,
this means that the representations provided by the model are encoding
the relevant information. In our case, we want to ask questions about
word properties to a model that has been trained at the character
level. In order to do that, we let the model read each training/test
word character-by-character, and we take the state of the model hidden
layer after having processed the last character in the word to be the
model's implicit representation of the word, on which we train the
diagnostic classifier. %
% Besides being sensitive, to some extent, to word boundaries, does
% the CNLM also store linguistic properties of words, such as their part
% of speech and number?
The experiments focus on German and Italian, as it's harder to design
reliable test sets for morphologically impoverished English.

\paragraph{Word classes (nouns vs.~verbs)}

For both German and Italian, we sampled 500 verbs and 500 nouns from
the Wikipedia training sets, requiring that they are unambiguously
tagged in the corpus. Since in many cases verbs and nouns are cued by
affixes, we selected the examples to have the same ending across the
two categories (\emph{-en} in German and \emph{-re} in Italian), so
that the models could not rely on affixes to disambiguate the part of
speech. We randomly selected 20 training examples (10 nouns and 10
verbs), and tested on the remaining items.  We repeated the experiment
100 times to control for random train-test split variation.
%  We recorded the final
% hidden state of a pre-trained CNLM after reading a word, without
% context, and trained a logistic noun-verb classifier on these
% representations.

While we controlled for the final affix as described above, it could
still be the case that other substrings inside words tend to cue verbs
or nouns. In order to assess to what extent the models are relying on
information gathered from broader contexts, we consider a baseline
that is only trained on word-internal information. This is a
character-level LSTM autoencoder trained to reconstruct words in
isolation.  The hidden state of the autoencoder should capture
discriminating orthographic features, but, by design, the latter has
no access to broader contexts.  We further considered word embeddings
from the output layer of the WordNLM. Unlike the character-based
models, the WordNLM cannot make educated guesses about words that are
not in its training vocabulary. These OOV words are by construction
less frequent, and thus likely to be in general more difficult. To get
a sense of both best-case-scenario and more realistic WordNLM
performance, we report its accuracy both excluding and including OOV
items (WordNLM$_{\textit{subs.}}$ and WordNLM in Table
\ref{tab:pos-results}, respectively). In the latter case, we let the
model make a random guess for OOV items.

Results are shown in Table~\ref{tab:pos-results}.  All language models
outperform the autoencoders, showing that they learned categories
based on broader distributional evidence, not just typical strings
cuing nouns and verbs. Moreover, the LSTM CNLM outperforms the RNN,
probably because it can track broader contexts. Not surprisingly, the
word-based model fares better on in-vocabulary words, but the gap,
especially in Italian, is rather narrow, and there is a strong
negative impact of OOV
words. % Figure~\ref{fig:pos-induction} shows how German performance evolves as the training set grows from 2 to 100 examples (Italian results are qualitatively identical). The CNLMs already distinguish the categories well with small training sets, while the autoencoder does not catch up even with 100 training examples per category.

\begin{table}[t]
%  \begin{small}
\footnotesize
    \begin{center}
      \begin{tabular}{l|l|l}
        &\emph{German}&\emph{Italian}\\
        \hline
        LSTM & 89.0 ($\pm$ 0.14) & 95.0 ($\pm$ 0.10) \\
        RNN & 82.0 ($\pm$ 0.64) & 91.9 ($\pm$ 0.24) \\
        Autoencoder & 65.1 ($\pm$ 0.22) & 82.8 ($\pm$ 0.26) \\
	      WordNLM$_{\textit{subs.}}$ & 97.4 ($\pm$ 0.05) & 96.0 ($\pm$ 0.06) \\
	      WordNLM & 53.5 ($\pm$ 0.18)  & 62.5 ($\pm$ 0.26) \\
      \end{tabular}
    \end{center}
 % \end{small}
	\caption{\label{tab:pos-results} Word class accuracy, with standard errors across 100 random train-test splits. `\emph{subs.}' marks in-vocabulary subset evaluation.} % (20 training examples)
\end{table}


% \begin{figure}
% \includegraphics[width=0.48\textwidth]{figures/german_pos_nouns_verbs.pdf}
% 	\caption{Word class accuracy as a function of training examples (German). }\label{fig:pos-induction}
% 	% \textbf{Please rename LM LSTM, Baseline Autoencoder and Words WordNLM}
% \end{figure}





\paragraph{Number}
We turn next to number, a more granular morphological feature. We
study German, as it possesses a rich system of nominal classes forming
plural through different morphological processes. We train a number
classifier on a subset of these classes, and test on the others. If a
model generalizes correctly, it means that it is sensitive to number
as an abstract feature, independently of its surface expression.

We extracted plural nouns from the German UD
treebank \cite{de2006generating,mcdonald2013universal}.  We selected
nouns with plurals in -\emph{n}, -\emph{s}, or -\emph{e} to train the classifier (e.g., \emph{Geschichte-n} `stories'), and tested on plurals formed with
-\emph{r} or through vowel change (\emph{Umlaut}, e.g., \emph{T{\"o}chter} for singular \emph{Tochter} `daughter').
%\textbf{Add one
%  example from training, one from testing.}

For the training set, we randomly selected 15 singulars and plurals
from each training class.  As plural suffixes make words longer, we
sampled singulars and
plurals %used rejection sampling \textbf{(cite something?)}
from a single distribution over lengths, to ensure that their lengths
were approximately matched.  For the test set, we selected all plurals
in -\emph{r} (127) or Umlaut (38), with their respective
singulars. %\textbf{(how many?)}.
We also used all remaining plurals ending in -\emph{n} (1467), -\emph{s} (98) and -\emph{e} (832) as in-domain test data.
%\textbf{Need information on the test set for the training classes.}
To control for the impact of training sample selection, we
report accuracies averaged over 200 random train-test splits.  We extract word
representations as above, and we compare the same models. %
%, excluding OOVs when testing the latter.
Results are summarized in Table \ref{tab:number-results}.

\begin{table}[t]
	\footnotesize
  \begin{center}
    \begin{tabular}{@{\hspace{0.3em}}l@{\hspace{0.42em}}|@{\hspace{0.42em}}c@{\hspace{0.45em}}|@{\hspace{0.45em}}l@{\hspace{0.65em}}l@{\hspace{0.15em}}}
      &train classes&\multicolumn{2}{c}{test classes}\\
      &\emph{-n/-s/-e}&\multicolumn{1}{c}{\emph{-r}}&\multicolumn{1}{c}{\emph{Umlaut}}\\      \hline
	    LSTM& 77.9 ($\pm$ 0.8) & 88.2 ($\pm$ 0.3) & 52.8 ($\pm$ 0.6) \\
	    RNN& 70.3 ($\pm$ 0.9) & 81.3 ($\pm$ 0.7) & 53.3 ($\pm$ 0.6)\\
	    Autoencoder& 64.0 ($\pm$ 1.0) & 73.8 ($\pm$ 0.6) & 59.2 ($\pm$ 0.5)\\
	    WordNLM$_{\textit{subs.}}$& 97.8 ($\pm$ 0.3) & 86.6 ($\pm$ 0.2) & 96.7 ($\pm$ 0.2)  \\ 
	    WordNLM & 82.1 ($\pm$ 0.1) & 73.1 ($\pm$ 0.1) & 77.6 ($\pm$ 0.1)  \\ 
    \end{tabular}
  \end{center}
  \caption{\label{tab:number-results} German number classification
    accuracy, with standard errors computed from 200 random train-test
    splits.  `\emph{subs.}' marks in-vocabulary subset evaluation.}
\end{table}


% \begin{table}[t]
% 	\footnotesize
%   \begin{center}
%     \begin{tabular}{l|c|l|lllllll}
%       &train classes&\multicolumn{2}{c}{test classes}\\
%       &\emph{-n/-s/-e}&\emph{-r}&\emph{Umlaut}\\      \hline
% 	    LSTM& 77.9 ($\pm$ 0.8) & \textbf{88.2} ($\pm$ 0.3) & 52.8 ($\pm$ 0.6) \\
% 	    RNN& 70.3 ($\pm$ 0.9) & 81.3 ($\pm$ 0.7) & 53.3 ($\pm$ 0.6)\\
% 	    Autoencoder& 64.0 ($\pm$ 1.0) & 73.8 ($\pm$ 0.6) & 59.2 ($\pm$ 0.5)\\
% 	    WordNLM& \textbf{97.8 ($\pm$ 0.3)} & 86.6 ($\pm$ 0.2) & \textbf{96.7} ($\pm$ 0.2)  \\ % when including OOVs: 81.0/83.8/81.5 & 72.9 & 77.6\\
%     \end{tabular}
%   \end{center}
% 	\caption{\label{tab:number-results} German number classification accuracy, with standard errors computed from 200 runs.}
% \end{table}



%\begin{table}[t]
%  \begin{center}
%    \begin{tabular}{l|l|l|lllllll}
%      &train classes&\multicolumn{2}{|c}{test classes}\\
%      &\emph{-n/-s/-e}&\emph{-r}&\emph{Umlaut}\\      \hline
%	    LSTM& 77.9 ($\pm$ 0.76) & \textbf{88.2} ($\pm$ 0.32) & 52.8 ($\pm$ 0.57) \\
%	    RNN& 70.3 ($\pm$ 0.88) & 81.3 ($\pm$ 0.72) & 53.3 ($\pm$ 0.64)\\
%	    Autoencoder& 64.0 ($\pm$ 0.96) & 73.8 ($\pm$ 0.61) & 59.2 ($\pm$ 0.47)\\
%	    WordNLM& \textbf{97.8 ($\pm$ 0.26)} & 86.6 ($\pm$ 0.15) & \textbf{96.7} ($\pm$ 0.19)  \\ % when including OOVs: 81.0/83.8/81.5 & 72.9 & 77.6\\
%    \end{tabular}
%  \end{center}
%	\caption{\label{tab:number-results} German number classification accuracy, with standard errors computed from 200 runs.}
%\end{table}
%



%\begin{table*}[t]
%  \begin{center}
%    \begin{tabular}{l|l|l|l}
%      &\multicolumn{1}{c}{train classes}&\multicolumn{2}{|c}{test classes}\\
%      &\multicolumn{1}{c}{\emph{-n/-s/-e}}&\emph{-r}&\emph{Umlaut}\\      \hline
%      LSTM& 82.1 ($\pm$ 0.74)/68.0 ($\pm$ 0.87)/83.7 ($\pm$ 0.68)  & \textbf{88.2} ($\pm$ 0.32) & 52.8 ($\pm$ 0.57) \\
%      RNN& 77.6 ($\pm$ 0.81)/60.0 ($\pm$ 0.94)/73.4 ($\pm$ 0.89) & 81.3 ($\pm$ 0.72) & 53.3 ($\pm$ 0.64)\\
%      Autoencoder& 73.2 ($\pm$ 0.97)/54.5 ($\pm$ 0.93)/64.4 ($\pm$ 0.99) & 73.8 ($\pm$ 0.61) & 59.2 ($\pm$ 0.47)\\
%	    WordNLM& \textbf{97.1 ($\pm$ 0.31)/97.9 ($\pm$ 0.27)/98.5 ($\pm$ 0.20)} & 86.6 ($\pm$ 0.15) & \textbf{96.7} ($\pm$ 0.19)  \\ % when including OOVs: 81.0/83.8/81.5 & 72.9 & 77.6\\
%    \end{tabular}
%  \end{center}
%	\caption{\label{tab:number-results} German number classification accuracy, with standard errors computed from 200 runs.}
%\end{table*}

%python char-lm-ud-stationary-separate-bidir-with-spaces-probe-baseline-prediction-wiki-plurals-2-tests-RNN.py  --batchSize 256 --char_dropout_prob 0.01 --char_embedding_size 50 --char_noise_prob 0.0 --hidden_dim 2048 --language german --layer_num 2 --learning_rate 0.1 --nonlinearity tanh --load-from wiki-german-nospaces-bptt-rnn-237671415 --sequence_length 30 --weight_dropout_hidden 0.0 --weight_dropout_in 0.0
%python char-lm-ud-stationary-separate-bidir-with-spaces-probe-baseline-prediction-wiki-plurals-2-tests-words.py  --language german --batchSize 128 --char_embedding_size 200 --hidden_dim 1024 --layer_num 2 --weight_dropout_in 0.1 --weight_dropout_hidden 0.35 --char_dropout_prob 0.0 --char_noise_prob 0.01 --learning_rate 0.2 --load-from wiki-german-nospaces-bptt-words-966024846

The classifier based on word embeddings is the most successful,
outperforming in most cases the best CNLM even in the more cogent
OOV-inclusive evaluation. This confirms the common observation that
word embeddings reliably encode number \cite{Mikolov:etal:2013a}. The
CNLM results are more mixed. Again the LSTM variant outperforms the
RNN one, but neither has problems detecting number for new words in
the training paradigms. They are also able to detect number in a new
paradigm (\emph{-r} plurals), showing that they are, to some extent,
inducing an abstract notion of number that is not tied to specific
orthographic exponents. In these experiments, the CNLMs significantly
outperform the autoencoder, and the LSTM even does better than the
word model in \emph{-r}-class generalization.

In contrast, the CNLMs fail to generalize to Umlaut plurals, where
they are virtually at chance level, and \emph{worse} than the
autoencoder. The most obvious difference between this type and
\emph{-r} pluralization is that the latter is an affixation process,
whereas Umlaut pluralization changes the stressed vowel of the
stem. Thus, a cautious interpretation is that the CNLMs developed a
representation of plural that is sufficiently abstract to generalize
to different affixation processes than the ones seen in training,
but. Note that, as we control for length, we are confident that the
network is not simply relying on this superficial feature to detect
plurals (which would explain why affixation cases are
easier). Similarly, the network can't simply rely on consonants cuing
plurals, both because many singular forms also end with consonants, and
because one of the training classes had \emph{-e} as plural marker.
% Evidently, CNLM number encoding is not abstract enough to
% generalize across very different surface morphological processes
% (adding a suffix vs.~changing the root vowel).
