\section{Related work}
\label{sec:related}

\paragraph{Character-based neural language models} Character-level
RNN-based language models have received some attention in the last
decade because of their potential greater generality with respect to
word-based models, as well because, intuitively, they should be able
to use cues, such as morphological information, that word-based models
miss by design. Early studies such as \newcite{Mikolov:etal:2011},
\newcite{Sutskever:etal:2011} and \newcite{Graves:2014} established
that CNLMs are in general not as good at language modeling as their
word-based counterparts, but lag only slightly behind (note that
character-level sentence prediction involves a much larger search
space than predicting at the word
level). \newcite{Sutskever:etal:2011} and \newcite{Graves:2014}
presented informal qualitative analysis showing that the CNLMs are
learning basic linguistic properties of their input. The latter, who
trained LSTM models, also showed that CNLMs can keep track to some
extent of hierarchical structure, as showing by their ability to
correctly balance parentheses when generating text. Our aim here is to
understand to what extent CNLMs trained on unsegmented input learn
various linguistic constructs. This is different from much work in the
area, that has focused on \emph{character-aware} architectures, that
combine in various ways character- and word-level information, with
the goal to develop state-of-the-art language models, also effective
in morphologically rich languages \citep[see, e.g.,][and references
there]{Bojanowski:etal:2016,Kim:etal:2016,Gerz:etal:2018}. \textbf{Give
  example of difference between character-level and character-aware.}
Note that all earlier work, even when not assuming an \emph{a priori}
word vocabulary, trains on text with whitespace and other orthographic
word-boundary cues. There is work on (morpheme-level) segmentation
using character-level RNNs \cite{Kann:etal:2016}, but the emphasis
there is in optimizing segmentation, as opposed to our interest in
probing what the network implicitly learned about morphemes and other
units.


Earlier work by Jeffrey Elman is close in spirit to ours. In
particular, \newcite{Elman:1990} presented experiments on phonotactics
and word segmentation related to ours, but using small-size toy
input. More recently, \cite{Radford:etal:2017} study CNLMs (or, more
precisely, byte-level models) with focus on understanding their
properties, but they focus on sentiment analysis. Closest to us,
\cite{Alishahi:etal:2017} study the linguistic knowledge induced by a
neural network that receives unsegmented acoustic input. They use
however a considerably more complex architecture, trained on
multimodal data, and focus on the phonological level. \textbf{Discuss
\newcite{Godin:etal:2018}.}

\paragraph{Probing linguistic knowledge of neural language models} There is extensive work on probing the linguistic properties of
word-based neural language models, as well as more complex
architectures such as sequence-to-sequence systems, see, e.g.,
\newcite{Li:etal:2016,Linzen:etal:2016,Shi:etal:2016,Adi:etal:2017,Belinkov:etal:2017,Hupkes:etal:2017,Kadar:etal:2017,Li:etal:2017}. Closest
to us, \newcite{Sennrich:2017} tests character- and subword-unit-level
models used as components of a machine translation system on a variety
of grammatical tests. He concludes that current character-based
decoders generalize better to unseen words, but capture less
grammatical knowledge than subword units. Still, his character-based
systems lag only marginal behind the subword architectures on
grammatical tasks such as handling agreement and negation.






